% Options for packages loaded elsewhere
\PassOptionsToPackage{unicode}{hyperref}
\PassOptionsToPackage{hyphens}{url}
\PassOptionsToPackage{dvipsnames,svgnames,x11names}{xcolor}
%
\documentclass[
  authoryear,
  preprint,
  5p,
  onecolumn]{elsarticle}

\usepackage{amsmath,amssymb}
\usepackage{iftex}
\ifPDFTeX
  \usepackage[T1]{fontenc}
  \usepackage[utf8]{inputenc}
  \usepackage{textcomp} % provide euro and other symbols
\else % if luatex or xetex
  \usepackage{unicode-math}
  \defaultfontfeatures{Scale=MatchLowercase}
  \defaultfontfeatures[\rmfamily]{Ligatures=TeX,Scale=1}
\fi
\usepackage{lmodern}
\ifPDFTeX\else  
    % xetex/luatex font selection
\fi
% Use upquote if available, for straight quotes in verbatim environments
\IfFileExists{upquote.sty}{\usepackage{upquote}}{}
\IfFileExists{microtype.sty}{% use microtype if available
  \usepackage[]{microtype}
  \UseMicrotypeSet[protrusion]{basicmath} % disable protrusion for tt fonts
}{}
\makeatletter
\@ifundefined{KOMAClassName}{% if non-KOMA class
  \IfFileExists{parskip.sty}{%
    \usepackage{parskip}
  }{% else
    \setlength{\parindent}{0pt}
    \setlength{\parskip}{6pt plus 2pt minus 1pt}}
}{% if KOMA class
  \KOMAoptions{parskip=half}}
\makeatother
\usepackage{xcolor}
\setlength{\emergencystretch}{3em} % prevent overfull lines
\setcounter{secnumdepth}{5}
% Make \paragraph and \subparagraph free-standing
\makeatletter
\ifx\paragraph\undefined\else
  \let\oldparagraph\paragraph
  \renewcommand{\paragraph}{
    \@ifstar
      \xxxParagraphStar
      \xxxParagraphNoStar
  }
  \newcommand{\xxxParagraphStar}[1]{\oldparagraph*{#1}\mbox{}}
  \newcommand{\xxxParagraphNoStar}[1]{\oldparagraph{#1}\mbox{}}
\fi
\ifx\subparagraph\undefined\else
  \let\oldsubparagraph\subparagraph
  \renewcommand{\subparagraph}{
    \@ifstar
      \xxxSubParagraphStar
      \xxxSubParagraphNoStar
  }
  \newcommand{\xxxSubParagraphStar}[1]{\oldsubparagraph*{#1}\mbox{}}
  \newcommand{\xxxSubParagraphNoStar}[1]{\oldsubparagraph{#1}\mbox{}}
\fi
\makeatother


\providecommand{\tightlist}{%
  \setlength{\itemsep}{0pt}\setlength{\parskip}{0pt}}\usepackage{longtable,booktabs,array}
\usepackage{calc} % for calculating minipage widths
% Correct order of tables after \paragraph or \subparagraph
\usepackage{etoolbox}
\makeatletter
\patchcmd\longtable{\par}{\if@noskipsec\mbox{}\fi\par}{}{}
\makeatother
% Allow footnotes in longtable head/foot
\IfFileExists{footnotehyper.sty}{\usepackage{footnotehyper}}{\usepackage{footnote}}
\makesavenoteenv{longtable}
\usepackage{graphicx}
\makeatletter
\newsavebox\pandoc@box
\newcommand*\pandocbounded[1]{% scales image to fit in text height/width
  \sbox\pandoc@box{#1}%
  \Gscale@div\@tempa{\textheight}{\dimexpr\ht\pandoc@box+\dp\pandoc@box\relax}%
  \Gscale@div\@tempb{\linewidth}{\wd\pandoc@box}%
  \ifdim\@tempb\p@<\@tempa\p@\let\@tempa\@tempb\fi% select the smaller of both
  \ifdim\@tempa\p@<\p@\scalebox{\@tempa}{\usebox\pandoc@box}%
  \else\usebox{\pandoc@box}%
  \fi%
}
% Set default figure placement to htbp
\def\fps@figure{htbp}
\makeatother

\newpageafter{author}
\makeatletter
\@ifpackageloaded{caption}{}{\usepackage{caption}}
\AtBeginDocument{%
\ifdefined\contentsname
  \renewcommand*\contentsname{Table of contents}
\else
  \newcommand\contentsname{Table of contents}
\fi
\ifdefined\listfigurename
  \renewcommand*\listfigurename{List of Figures}
\else
  \newcommand\listfigurename{List of Figures}
\fi
\ifdefined\listtablename
  \renewcommand*\listtablename{List of Tables}
\else
  \newcommand\listtablename{List of Tables}
\fi
\ifdefined\figurename
  \renewcommand*\figurename{Figure}
\else
  \newcommand\figurename{Figure}
\fi
\ifdefined\tablename
  \renewcommand*\tablename{Table}
\else
  \newcommand\tablename{Table}
\fi
}
\@ifpackageloaded{float}{}{\usepackage{float}}
\floatstyle{ruled}
\@ifundefined{c@chapter}{\newfloat{codelisting}{h}{lop}}{\newfloat{codelisting}{h}{lop}[chapter]}
\floatname{codelisting}{Listing}
\newcommand*\listoflistings{\listof{codelisting}{List of Listings}}
\makeatother
\makeatletter
\makeatother
\makeatletter
\@ifpackageloaded{caption}{}{\usepackage{caption}}
\@ifpackageloaded{subcaption}{}{\usepackage{subcaption}}
\makeatother
\journal{Fisheries Research}

\usepackage[]{natbib}
\bibliographystyle{elsarticle-harv}
\usepackage{bookmark}

\IfFileExists{xurl.sty}{\usepackage{xurl}}{} % add URL line breaks if available
\urlstyle{same} % disable monospaced font for URLs
\hypersetup{
  pdftitle={Atlantic Bluefin Tuna Area Transition Matrices Estimated From Electronic Tagging and SatTagSim},
  colorlinks=true,
  linkcolor={blue},
  filecolor={Maroon},
  citecolor={Blue},
  urlcolor={Blue},
  pdfcreator={LaTeX via pandoc}}


\setlength{\parindent}{6pt}
\begin{document}

\begin{frontmatter}
\title{Atlantic Bluefin Tuna Area Transition Matrices Estimated From
Electronic Tagging and SatTagSim}
\author[1]{Benjamin Galuardi%
\corref{cor1}%
}
 \ead{(email){[}benjamin.galuardi@noaa.gov{]}} 
\author[1]{Steven X. Cadrin%
%
}

\author[2]{Igor Arregi%
%
}

\author[2]{Haritz Arrizabalaga%
%
}

\author[3]{Antonio DiNatale%
%
}

\author[4]{Craig Brown%
%
}

\author[4]{Matt Lauretta%
%
}

\author[5]{Molly Lutcavage%
%
}


\affiliation[1]{organization={School of Marine Science and Technology,
University of Massachusetts Dartmouth, Fairhaven, MA,
USA},,postcodesep={}}
\affiliation[2]{organization={AZTI Technalia, Marine Research Division,
Pasaia (Gipuzkoa), Spain},,postcodesep={}}
\affiliation[3]{organization={ICCAT Secretariat, Madrid,
Spain},,postcodesep={}}
\affiliation[4]{organization={Southeast Fisheries Science Center, NOAA
Fisheries, Miami, FL, USA},,postcodesep={}}
\affiliation[5]{organization={Large Pelagics Research Center,
Gloucester, MA, USA},,postcodesep={}}

\cortext[cor1]{Corresponding author}








        





\end{frontmatter}
    

\section{Summary}\label{summary}

We demonstrate use of a telemetry based method for simulating individual
based movements to produce transition matrices for movement inclusive
models. A custom R package (SatTagSim) was used on Atlantic bluefin tuna
electronic tagging data from the Large Pelagics Research Center (UMass
Boston), AZTI Technalia, the Grande Bluefin Year Program, and the
National Marine Fisheries Service (USA) to derive transition matrices
for an 11-box and 4-box model. Migration rate matrices were produced for
western Atlantic tagged fish \textgreater185cm, and \textless185 cm, as
well as for eastern tagged fish (all sizes). The estimates provided are
fishery independent and external to the operational and assessment
frameworks. The results are highly informative to the assessment by (1)
providing estimates of bluefin tuna movement for consideration as fixed
values, prior probabilities on movement, or for direct comparison with
estimates of movement from mixed stock assessment models (2) to
facilitate discussion of movement rate assumptions and the use of
electronic tagging data in various assessment models.

Keywords: bluefin tuna, movement, stock assessment, electronic tags,

\section{Introduction}\label{introduction}

The benchmark assessment of Atlantic Bluefin tuna (\emph{Thunnus
thynnus}) in 2017 offers a unique opportunity to take advantage of the
wealth of electronic tagging data collected in the past 20+ years. The
Grande Bluefin Year Program (GBYP), a project within the International
Commission for the Conservation of Atlantic Tunas (ICCAT), in particular
offers a source of tagging data originating in the eastern Atlantic that
was not previously available. Management strategy evaluation (MSE) is a
priority for Atlantic bluefin tuna management and is currently a funded
initiative within ICCAT \citep{carruthers2014}. Development of operating
models is a key component to MSE and can be a vector for utilizing
electronic tag data \citep{kerr2012a, kerr2016, carruthers2016}

There are several challenges in making these data usable in assessment
activities. The first is to summarize the data in a way that is readily
usable by models. Not all tagging data are equal in terms of length,
area, and season of deployment. Fish are often tagged opportunistically,
when and where they are accessible e.g. \citep[\citet{cermeño2015},
\citet{wilson2015}]{galuardi2012}. Last, tagging data are expensive to
collect and hold incredibly high scientific value. Allowing time for
individual programs to publish results often means data are not quickly
available for other purposes.

The solution to many of these issues is to simplify the manner in which
the data are represented, and summarize it in an easily usable format.
Empirical summaries by area and season offer the simplest method for
incorporating many different data sources without compromising
individuals information for other purposes \citep{carruthers2016a}.
Simple summaries do not, however, take into account the uncertainties in
the positions from which they were derived and do not deal effectively
with the biases of short tracks in these datasets. A more robust method
is to use individual based simulation, based on the tagging data, to
produce movement probability metrics by season and area
\citep{galuardi2014}. Here, we use several electronic tagging datasets,
from the eastern and western Atlantic (Figure 1), to produce seasonal
movement matrices for use in current and future operational and
assessment models of Atlantic bluefin tuna. We use the 11-box model
developed for the GBYP MSE \citep{carruthers2016a} as well as a
simplified 4-box model. Simulation and matrix derivations were carried
out using a custom package (SatTagSim) for the R statistical software
\citep{rcoreteam2024}.

\section{Methods}\label{methods}

To generate movement probability matrices, we used methods modified and
updated from \citep{galuardi2014}. Tagging data were available from
several sources including the Large Pelagics Research Center of the
University of Massachusetts Boston, AZTI Technalia, the GBYP and the
NOAA Southeast Fisheries Science Center (Table 1). We used a total of
499 tagged fish ranging from 66 to over 300 cm curved fork length (CFL).
These data were split into eastern and western areas releases, defined
by the 45\textsuperscript{o} W Longitudinal stock boundary. Western
tagged fish were further split into categories \textgreater= 185cm CFL
(n = 189) and \textless{} 185cm CFL (n = 113), for a total of three
groups. The size split in the western Atlantic is consistent with the
commercial size limit of the United States and represents an
approximately 7-8 year old fish \citep{restrepo2010}. Fish tagged in the
eastern Atlantic (n = 166) were often tagged underwater and were not
measured consistently. Therefore, this group could not be reliably
divided by size/age. We recognize there are several additional
institutions, with substantial electronic tagging data, which were not
included in this study due to availability at the time of the work. For
each group of tagged fish, monthly advection parameters (u and v, mean
and standard deviation) were calculated from the final estimated tracks
for the group for that month. Diffusive parameters (D, mean and standard
deviation) were calculated using either the diffusion parameter
estimated from a Kalman filter state space model, or the fixed values
used in the estimation. Fixing a diffusion parameter often occurs when
using the KFtrack family of estimation models
\citep{sibert2003, lam2010}, and is standard practice for CLS derived
tracks. These parameters were used in a correlated random simulation
scheme that is parameterized in terms of the advective and diffusive
parameters \citep[\citet{sibert2006},\citet{galuardi2014}]{sibert2003}.

For each group, 84 fish were `released' each month (1,008/year).
Starting points were determined by mapping the density of observations
each month, and randomly selecting 1\textsuperscript{o} grid cells in
proportion to the densities. In this manner, simulations were started in
all months and in all areas where observations took place, while
retaining the relative importance of the observations. Each track was
simulated for 720 days (\textasciitilde{} 2 years), with daily positions
being derived by sampling the parameters in the previous step, and
generating steps or a correlated random walk \citep{galuardi2014}. A sea
surface temperature climatology \citep{boyer2013} was used to keep
simulated fish both in the water and in suitable water masses. Suitable
temperature was determined by determining the 95\% intervals of tag
measured temperatures from the LPRC dataset for each month (Figure 2).
Sea surface temperature can show considerable inter-annual variability
which could affect habitat suitability for a particular year. However,
since we combined tagging data from all years, a climatology was both
more appropriate and more practical for the simulation framework. ETOPO1
bathymetric data \citep{amante2009} were combined with the World Ocean
Atlas 1o surface temperatures for each month to represent habitat
available for simulated fish. The bathymetry layer was modified to
reduce barrier effects of Florida, the Straits of Gibraltar and through
the Laurentian Channel. The tag measured temperature limits were applied
to these layers for each month, and temperatures out of range were
clipped. A moving time window was used as a further selection criterion
so that for a given location, at least two months out of a three month
window had to meet the temperature range criteria. This step was
intended to keep simulated fish out of the sub-arctic water in winter
and out of tropical water in the summer.

Simulated fish were assigned to an area occupied each day through a
geoprocessing overlay step. Both an 11-box and 4 box model were
considered (Figure 3). To determine occupancy and transition, the first
box occupied in each season was selected as the box occupied that
season. Seasonal transition probabilities were then calculated by
cross-tabulating the previous and currently assigned box for each fish
for each season and dividing by the total for each row. For rows where
no simulated fish occurred, a value of 1 was inserted in the diagonal
position for that row (i.e.~no movement from that area between seasons).
This preserves the matrices Markovian properties where the row-wise
probabilities must sum to 1.

To determine the mean and variance of each transition, the described
simulation was repeated 1000 times, for a total of 1,008,000 simulated
fish in each group. A 95\% confidence interval was approximated for each
matrix cell by calculating the 2.5\% and 97.5\% quantiles. The
simulations were done in parallel using a desktop computer with an
8-core Intel i7 3.4GHz processor. Each group run took \textasciitilde18
hours to complete. In prior efforts, without the replication for
variance, runs of 12,000 simulated fish took 15-20 minutes on a standard
laptop running 4 cores in parallel.

\section{Results}\label{results}

A total of six movement probability matrices were constructed for the
three groups and two spatial structures (Figures 4-9). Values in each
cell represent the mean and 95\% confidence intervals for the 1000 runs.
Cells where the confidence interval was extremely small are indicative
of either very low occurrences in that season/area, or., that there were
no movements to those, or a combination of these. Figures 6 and 9
demonstrates this for region 1 (the Gulf of Mexico), for fish tagged in
the eastern Atlantic, and Figures 4, 5, 7 and 8 show similar results for
simulations from western tagged fish in Mediterranean boxes. Parameters
derived from the tagging data and used for the simulations are found in
Tables 2-4.

\section{Discussion}\label{discussion}

Spatially explicit and operational and assessment models require use of
some form of movement matrix to either to assign or estimate probability
of movement between areas of interest. Using a simulation framework
(SatTagSim) has several advantages: 1) producing these matrices external
to the modeling framework should improve computational performance by
reducing the confounding between movement and mortality parameter
estimates within mixed stock models 2) use of simulations, instead of
empirical summaries, acts as a form of bootstrap and extends the utility
of the information available 3) summarizing movement parameters across
time allows the use of any spatial stratification. In the case of
Atlantic Bluefin tuna, these could be across stock boundaries, spawning
regions, and among fishery reporting regions. Finally, use of electronic
tagging data provides fishery independent estimates of fish movements.

This work is an updated version of work presented at the 2016 bluefin
data preparation meeting, and follows the preliminary work in Galuardi
et al.~(2014). Several caveats should be kept in mind with respect to
this method. First, the simulation method does not constrain fish to any
specific area. This is a fundamentally different decision than what was
used in MAST model \citep{taylor2011}, where fish in certain regions
were not allowed to transit to other regions (e.g.~the Gulf of
St.~Lawrence to the Mediterranean). SatTagSim only constrains movement
according to the movement parameters derived from tagging data and
subsequent track estimation. This has the advantage of freely allowing
for stochasticity in the simulation according to the data. Second, the
temperature constraints are biologically reasonable but are somewhat
subjective in application. SatTagSim allows the user to specify, within
the moving time window, the number of suitable months for surface
temperature (1 -- 3 months). This was included in the method to keep
fish out of areas that are typically beyond their thermal range, but can
also act as a barrier if, for example, the temperature climatology
changes quickly from one month to the next. The level of temporal detail
in the World Ocean Atlas allows only for a monthly time step. A more
continuous climatology may be worth consideration.

The most useful consideration in using tagging within a mixed stock
assessment is knowledge of the stock of origin of the tagged fish.
Determining stock of origin from tagging data is not easily accomplished
since many assumptions must be made about spawning site fidelity,
spawning frequency, maturity, condition, spawning area, etc.
Furthermore, the prevalence of short tracks (Figure 10) disallows many
of these tracks from consideration since they were not at liberty long
enough even to apply these assumptions. Therefore, we did not make
assumptions of stock origin in this work and classified fish based only
according to where they were tagged. Given current knowledge of mixing
rates from otolith work \citep[\emph{e.g.},][]{rooker2014, siskey2016},
it is reasonable to assume a majority of our fish tagged originated from
the stock primarily associated with the tagging region. A detailed
breakdown of size, location and season of tagged fish would be necessary
to have a better understanding of possible stock origin of tagged fish.

Future work should include simulations and movement matrices using only
fish that visited a known spawning area. The simplest version of this
could be assigning western stock origin to tagged fish that visited the
Gulf of Mexico and eastern origins to fish that visited, or were tagged
in, the Mediterranean. Recent work highlighting the prevalence of
spawning in the Slope Sea in the western Atlantic \citep{richardson2016}
could add additional complexity and stock assignment but, currently,
there is not a consensus to which stock Slope Sea spawners should be
assigned, or how this information may affect our understanding of stock
structure \citep{walter2016}. These results also suggest possible
alternate spatial configurations, such as moving the split between areas
8 and 9 (Eastern Atlantic areas) to the Straits of Gibraltar.

Summarizing electronic tagging data movement parameters based on area of
tagging is not analogous to stock assignment, but it does have the
effect of functionally including movement patterns from a minority
occurring stock within the majority stock within the variability of
those parameters. For example, in our approach, western tagged fish
behave similarly. Eastern origin fish that travel to the Med represent
increased variability in the parameter estimates for months they are in
the eastern Atlantic and Mediterranean. The number of Western tagged
fish that crossed the 45\textsuperscript{o} W line is low: 9/189 for
fish \textgreater=185cm (4.8\%), 5/113 for fish \textless{} 185 cm
(4.4\%).

An interesting and valuable exercise would be to approach the parameter
summarization as described above, and assign a probability of switching
parameter sets when moving into areas and during seasons where mixing is
more likely to occur. This would have the biological effect of schools
mixing and fish taking on the movement characteristics of other fish.
This probability could be informed by the matrices reported here as well
as from biological sampling
\citep[\emph{e.g.,}][]{rooker2014, graves2015, siskey2016}

Atlantic bluefin tuna are currently assessed using the VPA 2-box method
\citep{porch2003}, which is usually applied separately to the eastern
and western stocks (ICCAT, 2014). The current benchmark assessment
affords the opportunity to apply this method with mixing, and to compare
results against other movement inclusive methods. One essential question
as to the nature of the mixing between the two stocks is whether the two
stocks represent an overlap model, where fish return to natal spawning
grounds without deviation, or a diffusive model where fish, typically
from the larger stock (eastern) more or less permanently emigrate and
take on the characteristics of the western stock. The current
application of SatTagSim more closely represents a diffusive method, but
it could easily be extended to examine other scenarios.

Finally, future work may include use of additional tagging datasets to
produce a more complete picture of telemetry based movement. These could
include those from Stanford University, the Department of Fisheries and
Oceans (Canada), The University of Cadiz
\citep{abascal2016, aranda2013}, and Ifremer \citep{fromentin2013}.
Increased data will refine these simulations by providing a more robust
analysis of bluefin migrations.

\subsection{Acknowledgements}\label{acknowledgements}

We thank Dr.~Tim Miller of the Northeast Fisheries Science Center,
Dr.~Lisa Kerr of the Gulf of Maine Research Institute, Dr.~Geoff Cowles
of the University of Massachusetts Dartmouth, and Dr.~Chi Hin (Tim) Lam
of University of Massachusetts, Boston, for valuable feedback, technical
assistance and inspiration. This work was possible due to the
NOAA/Seagrant Population Dynamics Fellowship (NA11OAR4170184) to B.
Galuardi and a Bluefin Tuna Research Program grant (NA13NMF4720061) to
M. Lutcavage. Special thanks to GBYP and tagging partners for tagging
data: Fundacion AZTI, Instituto Espanol de Oceanografia, INRH, IZOR,
COMBIOMA, Istanbul University, Universitat Politecnica de Valencia, Kali
Tuna (Croatia) and the World Wildlife Fund.

This work was carried out under the provision of the ICCAT Atlantic Wide
Research Programme for Bluefin Tuna (GBYP), funded by the European
Union, by several ICCAT CPCs, the ICCAT Secretariat and by other
entities (see: http://www.iccat.int/GBYP/en/Budget.htm). The contents of
this paper do not necessarily reflect the point of view of ICCAT or of
the other funders, which have not responsibility about them, neither do
they necessarily reflect the views of the funders and in no ways
anticipate the Commission's future policy in this area.  

\subsection{References}\label{references}

\renewcommand{\bibsection}{}
\bibliography{references.bib}

  Table 1. Datasets used in simulations.

Owner Primary Area of release Size at release Number of tags Years
tagged Principal Investigator LPRC/UMass Boston Canadian Maritimes, Gulf
of Maine 75-300 cm 302 1997-2014 Molly Lutcavage NOAA, SEFSC Gulf of
Mexico 190-270 cm 31 2010-2012 Craig Brown AZTI Bay of Biscay 66-110 cm
20 2005-2010 Igor Arregi/Haritz Arrizabalaga GBYP Multiple; Eastern
Atlantic and Med 56-265 cm 146 2011-2015 Antonio Di Natale Total Western
333\\
Total Eastern 166\\
Grand Total 499

Table 2. Parameters derived from 189 western tagged Atlantic bluefin
\textgreater= 185 cm CFL. Advective terms are u and v while D indicates
diffusion. The `sd' prefix indicates standard deviation. Units are
nautical miles/day (nmd-1) for u, v, sd.u, sd.v, and nautical miles
squared per day (nm2d-1) for D and sd.D.

Month u v D sd.u sd.v sd.D 1 2.14 -2.04 1221.28 12.30 8.07 1253.39 2
-1.24 -1.69 1304.15 18.18 7.93 1300.40 3 -2.27 0.61 1405.27 16.04 11.71
1337.40 4 2.03 5.70 1284.55 16.73 10.32 1299.74 5 9.26 1.07 1282.14
16.32 10.08 1386.94 6 13.60 17.15 1388.65 21.19 17.73 1470.72 7 1.23
4.27 1253.62 14.63 13.26 1449.20 8 0.57 -0.62 1053.34 4.55 5.36 1365.82
9 0.88 -0.22 1062.58 4.75 9.36 1315.52 10 -2.99 -7.69 1081.05 10.01
11.06 1178.61 11 -8.16 -9.05 1076.06 10.19 9.39 1135.88 12 -1.95 -2.10
1128.35 9.33 6.71 1165.03

  Table 3. Parameters derived from 113 western tagged Atlantic bluefin
\textless{} 185cm CFL. Terms and units follow those in Table 2.

Month u v D sd.u sd.v sd.D\\
1 1.39 -1.92 961.81 10.96 8.38 1056.34 2 0.40 -0.88 945.05 13.18 6.28
1064.86 3 0.24 -0.14 955.22 12.75 9.40 1061.66 4 1.18 5.58 923.52 13.24
8.91 1043.36 5 0.76 2.37 943.17 11.61 6.38 1152.37 6 1.32 3.32 1085.69
15.72 10.75 1288.33 7 -0.32 1.22 981.51 10.74 7.47 1193.80 8 0.33 -0.53
848.29 3.42 4.76 1143.32 9 0.32 -1.08 903.56 4.44 6.53 1067.70 10 -2.42
-5.62 937.12 7.88 9.36 1000.87 11 -5.91 -7.21 947.13 10.26 10.01
997.11\\
12 -1.19 -2.83 936.52 7.97 7.75 992.68

Table 4. Parameters derived from 166 eastern tagged Atlantic bluefin.
Terms and units follow those in Table 2.

Month u v D sd.u sd.v sd.D 1 -0.89 -1.23 946.54 10.34 4.98 718.32 2
-1.23 -0.21 845.77 6.77 4.18 574.34 3 -0.77 -0.82 790.73 4.75 2.88
345.45 4 0.53 0.76 790.98 4.34 2.03 352.88 5 6.44 1.54 970.84 25.05 7.89
493.53 6 4.48 0.91 879.66 19.11 7.15 478.84 7 -8.81 4.01 915.88 15.46
11.16 514.00 8 -0.88 3.12 1010.16 5.17 8.06 572.07 9 -1.77 1.49 1017.43
6.88 8.07 757.13 10 0.17 -0.29 1027.69 7.05 6.01 724.73 11 -1.45 0.11
1037.25 5.69 4.07 699.24 12 -0.69 -0.71 1018.30 6.28 6.90 704.73

Figure 1. Electronic tagging data used in the analysis. Large Pelagics
Research Center (LPRC) included fish both \textgreater=185cm and
\textless{} 185cm and included both immature and mature individuals. The
NOAA data were all mature fish, \textgreater=185cm. AZTI data were
mostly immature fish (Arregi at al., in prep). GBYP likely had a mix of
mature and immature fish.

Figure 2. Cumulative distributions of tag measured temperatures from the
LPRC electronic tagging data (n = 341 fish). Plots show the percentage
of observations (y-axis) that fell with each temperature (x-axis). The
mean (red line) and 95\% quantiles (blue lines) are displayed for
reference. The temperature range used for the simulations is less
stringent than these observations to account for unobserved variance.

Figure 3. Two spatial stratification schemes considered for movement
transitions. The 4-box model is an aggregated subset of the 11-box
model. The 4-box model was adjusted slightly so that the western
boundary did not cross into the Pacific Ocean.

Figure 4. Western tagged fish \textgreater= 185 cm CFL, 4 box model. In
each cell, the first line represents the mean transition probability
from the previous season to the current season. The colorbar represents
the mean as a visual aid. The second line in each cell is the 95\%
confidence interval of the 1000 replicate runs. Figures 5-9 are labeled
in this manner.  

Figure 5. Western tagged fish \textless{} 185 cm CFL, 4 box model.  

Figure 6. Eastern tagged fish, 4 box model.

Figure 7. Western tagged fish \textgreater= 185 cm CFL, 11 box model

Figure 8. Western tagged fish \textless{} 185 cm CFL, 11 box model.

Figure 9. Eastern tagged fish, 11 box model.  

Figure 10. Histogram of days at liberty for the combined NOAA and LPRC
datasets (n = 333). This highlights the prevalence of relatively short
tagging durations.





\end{document}
